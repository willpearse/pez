\documentclass{bioinfo}
\copyrightyear{2015}
\pubyear{2015}
\begin{document}
\firstpage{1}

\title[\emph{pez}]{\emph{pez}: Phylogenetics for the Environmental
  Sciences} \author[Pearse \textit{et~al}]{William D.\
  Pearse\,$^{1,2,3}$\footnote{to whom correspondence should be
    addressed}, Marc Cadotte\,$^{4}$, Jeannine Cavender-Bares\,$^1$,
  Anthony R.\ Ives$^5$, Caroline Tucker\,$^6$, Steve Walker\,$^7$ and
  Matthew R.\ Helmus\,$^8$} \address{$^1$Department of Ecology,
  Evolution, and Behavior, University of Minnesota, 100 Ecology
  Building, 1987 Upper Buford Circle, Saint Paul, Minnesota, USA,
  $^2$Department of Biology, McGill University, 1205 Avenue Docteur
  Penfield, Montr\'{e}al, Qu\'{e}bec, Canada, $^3$D\'{e}partement des
  Sciences Biologiques, Universit\'{e} du Qu\'{e}bec \`{a}
  Montr\'{e}al, Succursale Centre-ville, Montr\'{e}al, Qu\'{e}bec,
  Canada, $^4$Department of Biological Sciences, University of
  Toronto–Scarborough, 1265 Military Trail, Scarborough, Ontario,
  Canada, $^5$Department of Zoology, University of Wisconsin, Madison,
  Wisconsin, USA $^6$Department of Ecology and Evolutionary Biology,
  University of Colorado, Boulder, CO, USA, $^7$Department of
  Mathematics and Statistics, McMaster University, Hamilton, Ontario,
  Canada, $^8$Department of Animal Ecology, Vrije Universiteit, 1081
  HV, Amsterdam, The Netherlands} \history{} \editor{}
\maketitle
\begin{abstract}
\section{Summary:}
\emph{pez} is an \emph{R} package that permits measurement, modelling,
and simulation of phylogenetic structure in ecological
data. \emph{pez} contains the first implementation of many methods in
\emph{R}, and aggregates existing data structures and methods into a
single, coherent package.
\section{Availability:}
\emph{pez} is released under the GPL v3 open-source license, available
on the Internet from CRAN
(\href{http://cran.r-project.org}{http://cran.r-project.org}). The
package is under active development, and the authors welcome
contributions (see
\href{http://github.com/willpearse/pez}{http://github.com/willpearse/pez}).
\section{Contact:} \href{will.pearse@gmail.com}{William D.\ Pearse; will.pearse@gmail.com}
\end{abstract}
\section{Introduction}
Community phylogenetics (or eco-phylogenetics) combines ecology and
evolutionary biology, linking ecological phenomena with the
evolutionary processes that generate species and their traits
\citep[see][]{Webb2002}. This growing field has produced a number of
statistical tools and software code to implement them
\citep[\emph{e.g.},][]{Kembel2010}, but this code is disparate and
handles data differently, making routine data analyses challenging. In
many cases, published methods are not formally implemented in a
software package and are available only as supplementary materials to
papers. Without active (public) maintenance, these valuable techniques
are effectively lost to the scientific community.

Here, we present the \emph{pez} \emph{R} \citep{R2015} package which
provides an intuitive framework that merges existing tools and
supports the development of the next generation of eco-phylogenetic
methods. It is based on an \emph{R} class (\emph{comparative.comm})
that links phylogenetic, community, environmental, and trait data in a
single object. With a \emph{comparative.comm} object, one can
presently calculate over 30 phylogenetic and trait diversity metrics
(described in the documentation for \emph{pez.metrics}), including
more than 10 that were previously unavailable in \emph{R}. \emph{pez}
additionally implements statistical models
\citep{Cavender-Bares2004,Ives2011,Rafferty2013} to infer the
processes that underlie patterns in community phylogenetic and trait
data. Users can also perform simulations under various community
assembly and evolutionary processes, to help test hypotheses about the
structure of biodiversity.
\section{Description}
\subsection{Data manipulation and storage}
\emph{pez} provides a unified class to contain eco-phylogenetic data,
and provides wrapper functions to manipulate the species and site
composition of community, trait, environmental, and phylogenetic
data. Using the \emph{phy.build} family of functions, users can
generate phylogenies for their dataset \citep[making use of methods
previously implemented in \emph{phyloGenerator};][]{Pearse2013}.
\emph{pez} integrates and makes use of much existing \emph{R} code
\citep{Bortolussi2012,Genz2013,Koenker2015,Labierte2014,Oksanen2015,Paradis2004},
and its \emph{comparative.comm} class is directly compatible with all
\emph{caper} \citep{Orme2013} code, easing comparative analysis of
species trait data.

\subsection{Metrics}
Following the classification of \citet{Pearse2014review}, \emph{pez}
simplifies the calculation and comparison of over 30 metrics by
grouping them into four categories: \emph{shape}, \emph{evenness},
\emph{dispersion}, and \emph{dissimilarity}. Shape metrics measure the
structure of a community phylogeny, while evenness metrics
additionally incorporate species abundances. Dispersion metrics are
used to examine whether phylogenetic biodiversity in an assemblage
differs from the expectation of random assembly from a given set of
species---a species pool. Finally, dissimilarity metrics measure the
pairwise difference in phylogenetic biodiversity between
assemblages. This classification links directly onto the kind of data
the investigator has at hand: assemblage occupancy (\emph{shape}),
assemblage abundance (\emph{evenness}), source pool
(\emph{dispersion}), and assemblage comparisons
(\emph{dissimilarity}).

Other packages \citep[notably \emph{picante};][]{Kembel2010} allow for
the calculation of some of the metrics contained within
\emph{pez}. \emph{pez} unifies these metrics with other metrics not
previously implemented under a common interface that can be extended
to trait and transformed phylogenetic matrices
\citep[\emph{sensu}][]{Letten2014}. For example, \emph{pez} provides
the first implementation of the \emph{traitgram} framework
\citep{Ackerly2009,Cadotte2013}, allowing direct comparison of the
extent of phylogenetic and functional trait community
structure. \emph{pez} also contains a flexible metric estimation and
null-model generation suite (the \emph{generic.metric} family), that
can be used to easily compare different metrics and perhaps generate
new \emph{dispersion} metrics. New null models based around species'
trait values (using \emph{trait.asm}), and expected mean pairwise
phylogenetic and trait distances (using \emph{ConDivSim}) can also be
simulated. The speed and ease by which pez allows for the user to
compare many metrics is important because different metrics reveal
separate aspects of eco-phylogenetic structure \citep{Cadotte2010}.
\subsection{Statistical Models}
\emph{pez} implements the \citet{Cavender-Bares2004} regression
framework (\emph{fingerprint.regression}), and extends it to include
more metrics of trait evolution. While community phylogenetics has
been criticised for relying on the assumption of niche conservatism to
explain ecological assembly \citep[\emph{e.g.},][]{Mayfield2010}, this
early (yet previously unavailable) approach does not rely on niche
conservatism. Instead, it regresses the correlation between species'
trait similarity and observed degree of co-occurrence against summary
statistics of the evolution of those traits. The approach goes beyond
simply describing the phylogenetic or ecological structure of an
assemblage, and instead draws links between how traits have evolved
and how they play-out in their present-day ecological context.

The \emph{pglmm} family of functions permit regression modelling of
community \citep{Ives2011} and interaction network data
\citep{Rafferty2013} using the Phylogenetic Generalised Linear Mixed
Model (PGLMM) framework. PGLMMs permit tests of the mechanisms that
underly ecological and evolutionary structure in multiple
communities. By fitting random effect terms that account for
phylogenetic and trait co-variance, the user can statistically
estimate how likely similar and dissimilar species are to (co-) occur
as a function of their traits, environmental conditions, and the
presence of interacting species.
\subsection{Simulations}
Hypothesis testing and the development of new eco-phylogenetic methods
requires simulation of data under known conditions, and \emph{pez}
contains functions to facilitate this. The \emph{scape} family of
functions \citep[following][]{Helmus2012} simulate species'
distributions across environmental gradients. Species' environmental
tolerances, range size, and negative species interactions can be
simulated to contain phylogenetic signal and generate patterns of
phylogenetic structure at different spatial scales. The
\emph{sim.meta.phy.comm} family of functions simulate phylogeny, trait
evolution, and assemblage structure simultaneously across a landscape.
Species' abundances across the landscape are proportional to the fit
of each species' trait value to the conditions in each assemblage, and
individuals migrate at random across the landscape. The
\emph{sim.meta.phy.comm} functions do not require the user to specify
a phylogeny, and combined with the \emph{phy.sim} family, provide a
toolkit for the development of new simulation scenarios.
\section*{Acknowledgements}
F.\ Boivin, E.\ Lind, B.\ Waring, two anonymous reviewers, and M.\
Pennell provided useful feedback.

\begin{thebibliography}{30}
\providecommand{\natexlab}[1]{#1}
\providecommand{\url}[1]{\texttt{#1}}
\providecommand{\urlprefix}{URL }

\bibitem[{Ackerly(2009)}]{Ackerly2009} Ackerly, D. (2009) Conservatism
  and diversification of plant functional traits: Evolutionary rates
  versus phylogenetic signal \emph{Proceedings of the National Academy
    of Sciences} \textbf{106}, 19699-19706.

\bibitem[{Bortolussi \emph{et~al.}(2012)Bortolussi, Durand, Blum \&
    Francois}]{Bortolussi2012} Bortolussi, N., Durand, E., Blum, M. \&
  Francois, O. (2012) \emph{apTreeshape: Analyses of Phylogenetic
    Treeshape}. R package version 1.4-5. 

\bibitem[{Cadotte \emph{et~al.}(2013)Cadotte, Albert \& Walker}]{Cadotte2013}
Cadotte, M., Albert, C.H. \& Walker, S.C. (2013) The ecology of differences:
  assessing community assembly with trait and evolutionary distances.
  \emph{Ecology Letters} \textbf{16}, 1234--1244.

\bibitem[{Cadotte \emph{et~al.}(2010)Cadotte, Davies, Regetz, Kembel, Cleland
  \& Oakley}]{Cadotte2010}
Cadotte, M.W., Davies, T.J., Regetz, J., Kembel, S.W., Cleland, E. \& Oakley,
  T.H. (2010) Phylogenetic diversity metrics for ecological communities:
  integrating species richness, abundance and evolutionary history.
  \emph{Ecology Letters} \textbf{13}, 96--105.

\bibitem[{Cavender-Bares \emph{et~al.}(2004)}]{Cavender-Bares2004}
  Cavender-Bares, J., Ackerly, D.D., Baum, D.A. \& Bazzaz, F.A. (2004)
  {Phylogenetic overdispersion in Floridian oak
    communities}. \emph{The American Naturalist} \textbf{163},
  823--43.

\bibitem[{Dray \& DuFour (2007)Dray \& Dufor}]{Dray2007} Dray, S. \&
  Dufour, A.B. (2007) The ade4 package: implementing the duality
  diagram for ecologists. Journal of Statistical Software. 22(4):
  1--20.

\bibitem[{Genz \emph{et~al.}(2013)}]{Genz2013} Genz, A., Bretz, F.,
  Miwa, T., Mi, X., Leisch, F., Scheipl, F., Hothorn, T. (2013)
  \emph{mvtnorm: Multivariate Normal and t
    Distributions}. R package version 1.0-2.

\bibitem[{Helmus \& Ives(2012)}]{Helmus2012}
Helmus, M.R. \& Ives, A.R. (2012) Phylogenetic diversity-area curves.
  \emph{Ecology} \textbf{93}, S31--S43.

\bibitem[{Ives \& Helmus(2011)}]{Ives2011}Ives, A.R. \&
  Helmus, M.R. (2011) Generalized linear mixed models for phylogenetic
  analyses of community structure \emph{Ecological Monographs}
  \textbf{81}, 511--525

\bibitem[{Kembel \emph{et~al.}(2010)Kembel, Cowan, Helmus, Cornwell, Morlon,
  Ackerly, Blomberg \& Webb}]{Kembel2010}
Kembel, S.W., Cowan, P.D., Helmus, M.R., Cornwell, W.K., Morlon, H., Ackerly,
  D.D., Blomberg, S.P. \& Webb, C.O. (2010) Picante: R tools for integrating
  phylogenies and ecology. \emph{Bioinformatics} \textbf{26}, 1463--1464.

\bibitem[{Koenker(2015)}]{Koenker2015} Koenker, R. (2015)
  \emph{quantreg: Quantile
    Regression}. R package version 5.11.

\bibitem[{Labiert\'{e} \emph{et~al.}(2014)}]{Labierte2014} Labiert\'{e}, E., Legendre,
  P. \& Shipley, B. (2014) \emph{FD: measuring functional diversity
    from multiple traits, and other tools for functional
    ecology}. R package version 1.0-12.

\bibitem[{Letten \& Cornwell(in~press)}]{Letten2014} Letten, A.D. \&
  Cornwell, W.K. (in press) Trees, branches and (square) roots: why
  evolutionary relatedness is not linearly related to functional
  distance. \emph{Methods in Ecology \& Evolution}.

\bibitem[{Mayfield \& Levine(2010)}]{Mayfield2010}
  Mayfield, M.M. \& Levine, J.M. (2010) Opposing effects of
  competitive exclusion on the phylogenetic structure of communities
  \emph{Ecology Letters} \textbf{13} 1085--1093.

\bibitem[{Oksanen \emph{et~al.}(2015)Oksanen, Blanchet, Kindt,
  Legendre, Minchin, O'Hara, Simpson, Solymos, Stevens,
  Wagner}]{Oksanen2015} Oksanen, J., Guillaume, Blanchet, F.G., Kindt,
R., Legendre, P., Minchin, P.R., O'Hara, R.B., Simpson, G.L., Solymos,
P., Stevens, M.H.H. \& Wagner, H. (2015) \emph{vegan: Community Ecology
  Package}. R package version 2.2-1.

\bibitem[{Orme \emph{et~al.}(2013)Orme, Freckleton, Thomas, Petzoldt, Fritz,
  Isaac \& Pearse}]{Orme2013}
Orme, D., Freckleton, R., Thomas, G., Petzoldt, T., Fritz, S., Isaac, N. \&
  Pearse, W.D. (2013) \emph{caper: comparative analyses of phylogenetics and
  evolution in {R}}. R package version 0.5.2.

\bibitem[{Paradis \emph{et~al.}(2004)Paradis, Claude \&
    Strimmer}]{Paradis2004} Paradis, E., Claudem, J. \& Strimmer, K.
  (2004) APE: analyses of phylogenetics and evolution in R
  language. \emph{Bioinformatics} \textbf{20} 289--290.

\bibitem[{Pearse \& Purvis(2013)}]{Pearse2013} Pearse, W.D. \& Purvis,
  A. (2013) phyloGenerator: an automated phylogeny generation tool for
  ecologists. \emph{Methods in Ecology and Evolution} \textbf{7},
  692--698.

\bibitem[{Pearse \emph{et~al.}(2014)Pearse, Cavender-Bares, Puvis \&
  Helmus}]{Pearse2014review}
Pearse, W.D., Cavender-Bares, J., Puvis, A. \& Helmus, M.R. (2014) Metrics and
  models of community phylogenetics. \emph{Modern Phylogenetic Comparative
  Methods and their Application in Evolutionary Biology---Concepts and
  Practice} (ed. L.Z. Garamszegi), Springer-Verlag, Berlin, Heidelberg.

\bibitem[{{R Core Team}(2015)}]{R2015}
{R Core Team} (2015) \emph{R: A language and environment for statistical
  computing}. R Foundation for Statistical Computing, Vienna, Austria.

\bibitem[{Rafferty \& Ives(2013)}]{Rafferty2013} Rafferty, N.E. \&
  Ives, A.R. (2013) Phylogenetic trait-based analyses of ecological
  networks. \emph{Ecology} \textbf{94}, 2321--2333.

\bibitem[{Webb \emph{et~al.}(2002)Webb, Ackerly, McPeek \& Donoghue}]{Webb2002}
Webb, C.O., Ackerly, D.D., McPeek, M.A. \& Donoghue, M.J. (2002) Phylogenies
  and community ecology. \emph{Annual Review of Ecology and Systematics}
  \textbf{33}, 475--505.
\end{thebibliography}
\end{document}
%%% Local Variables:
%%% mode: latex
%%% TeX-master: t
%%% End:
