\documentclass{bioinfo}
\copyrightyear{2014}
\pubyear{2014}
\begin{document}
\firstpage{1}

\title[\emph{pez}]{\emph{pez}: Phylogenetics for the Environmental
  Sciences} \author[Pearse \textit{et~al}]{William D.\
  Pearse\,$^{1,2,3}$\footnote{to whom correspondence should be
    addressed}, Marc Cadotte\,$^{4}$, Jeannine Cavender-Bares\,$^1$,
  Anthony R.\ Ives$^5$, Caroline Tucker\,$^6$, Steve Walker\,$^7$ and
  Matthew R.\ Helmus\,$^8$} \address{$^1$Department of Ecology,
  Evolution, and Behavior, University of Minnesota, 100 Ecology
  Building, 1987 Upper Buford Circle, Saint Paul, Minnesota, USA,
  $^2$Department of Biology, McGill University, 1205 Avenue Docteur
  Penfield, Montr\'{e}al, Qu\'{e}bec, Canada, $^3$D\'{e}partement des
  Sciences Biologiques, Universit\'{e} du Qu\'{e}bec \`{a}
  Montr\'{e}al, Succursale Centre-ville, Montr\'{e}al, Qu\'{e}bec,
  Canada, $^4$Department of Biological Sciences, University of
  Toronto–Scarborough, 1265 Military Trail, Scarborough, Ontario,
  Canada, $^5$Department of Zoology, University of Wisconsin, Madison,
  Wisconsin, USA $^6$Department of Ecology and Evolutionary Biology,
  University of Colorado, Boulder, CO, USA, $^7$Department of
  Mathematics and Statistics, McMaster University, Hamilton, Ontario,
  Canada, $^8$Department of Animal Ecology, Vrije Universiteit, 1081
  HV, Amsterdam, The Netherlands} \history{} \editor{}
\maketitle
\begin{abstract}
\section{Summary:}
\emph{pez} is an \emph{R} package that calculates $>$30 community
phylogenetic metrics, statistically models phylogenetic, trait, and
community data, and simulates community structure. It provides a
common programmatic standard for the management and manipulation of
eco-phylogenetic data \emph{R}.
\section{Availability:}
\emph{pez} is released under the GPL v3 open-source license, available
on the Internet from CRAN
(\href{http://cran.r-project.org}{http://cran.r-project.org}). The
package is under active development, and the authors welcome
contributions (see
\href{http://github.com/willpearse/pez}{http://github.com/willpearse/pez}).
\section{Contact:} \href{will.pearse@gmail.com}{William D.\ Pearse; will.pearse@gmail.com}
\end{abstract}
\section{Introduction}
Community phylogenetics (or eco-phylogenetics) combines ecology and
evolutionary biology, linking ecological phenomena with the
evolutionary processes that generate species and their traits
\citep[see][]{Webb2002}. This growing field has produced a number of
statistical tools and software code to implement them
\citep[\emph{e.g.},][]{Kembel2010}, but the field's code is disparate
and handles data differently, making routine data analyses
challenging. In many cases, methods are not formally implemented in a
software package and are available only as supplementary materials to
papers. Without active (public) maintenance, these valuable techniques
are effectively lost to the scientific community.

\emph{pez} provides an \emph{R} \citep{R2015} class
(\emph{comparative.comm}) that integrates phylogenetic, community,
environmental, and trait data in a single object. Table
\ref{metricTable} gives an overview of its features. Using a
\emph{comparative.comm} object, one can calculate a large body of
phylogenetic biodiversity metrics, many of which were previously
unavailable in \emph{R}. These metrics are grouped according to the
framework outlined in \citet{Pearse2014review}, easing the
interpretation and comparison of different aspects of phylogenetic
biodiversity. Further, \emph{pez} implements several modelling
frameworks \citep{Cavender-Bares2004,Ives2011}, and provides functions
to simulate phylogenetic and ecological data. It is our hope that
\emph{pez} will provide a universal, easy to use, and extendable
\emph{R} framework for eco-phylogenetic analysis.
\section{Description}
\subsection{Data manipulation and storage}
\emph{pez} extends the data manipulation features of \emph{R}, and
provides a unified class to contain all eco-phyogenetic data. It
ensures that community, trait, environmental, and phylogenetic data
are compatible, and allows simultaneous manipulation of species and
site composition across all data types. \emph{pez} is compatible with
all \emph{caper} \citep{Orme2013} code, facilitating comparative
analysis of species trait data.

\subsection{Metrics}
Following the classification of \citet{Pearse2014review}, \emph{pez}
eases the calculation and comparison of over 30 metrics by grouping
them into four categories: \emph{shape}, \emph{evenness},
\emph{dispersion}, and \emph{dissimilarity}. Shape metrics measure the
structure of a community phylogeny, while evenness metrics
additionally incorporate species abundances. Dispersion metrics are
used to examine whether phylogenetic biodiversity in an assemblage
differs from the expectation of random assembly from a given set of
species---a species pool. Finally, dissimilarity metrics measure the
pairwise difference in phylogenetic biodiversity between
assemblages. The \emph{traitgram} framework
\citep{Ackerly2009,Cadotte2013}, which directly compares phylogenetic
and functional trait community structure, is also
implemented. Speeding and easing the comparison of community
phylogenetic metrics is important, since different metrics can reveal
separate aspects of eco-phylogenetic structure \citep{Cadotte2010}.

\begin{table*}
  \processtable{Overview of some of the functions available in \emph{pez}. * indicates a family of functions which are documented under this name in the package.\label{metricTable}}
  {\begin{tabular}{p{3cm} p{14cm}}\toprule
      Function(s) & Description\\\midrule
      \emph{comparative.comm} & Stores community, phylogenetic, environmental, and species trait data\\
      \emph{shape}, \emph{evenness}, \emph{dispersion}, \emph{dissimilarity} & Calculates metrics following the frameowkr of \citet{Pearse2014review}, including $E_{ED}$, $H_{ED}$, $H_{AED}$, \& $E_{AED}$ \citep{Cadotte2010}, Pagel's $\lambda$, $\delta$ \& $\kappa$ \citep{Pagel1999}, phylogenetic entropy \citep{Allen2009}, and \emph{INND} \citep{Ness2011}, all integrated with other metrics in existing packages. Permits use of trait distance matrices and \emph{traitgram} \citep{Ackerly2009,Cadotte2013} approaches, and square-root transformations \citep[following][]{Letten2014}.\\
      \emph{fingerprint.regression}, \emph{eco.xxx.regression}* & Compares phylogenetic, community co-existence, trait similarity, environmental, and environmental overlap matrices using Mantel tests, quantile regressions, or linear models, following \citet{Cavender-Bares2004} \\
      \emph{sim.phy}*, \emph{sim.meta}*, \emph{scape}, \emph{trait.asm}, \emph{ConDivSim} & Simulate phylogenies, assemblages, meta-communities, regional landscapes, and expected phylogenetic diversities. In particular, \emph{scape} simulates community phylogenetic structure across a landscape, simulating phylogenetic repulsion, attraction, niche width, and range size, following \citet{Helmus2012}\\
      \emph{phy.build}* & Tools for building assemblage phylogenies, following the \emph{merge}/\emph{replace} methods of \emph{phyloGenerator} \citep{Pearse2013}\\
      \emph{pglmm}* & Phylogenetic Generalised Linear Mixed Models of communty \citep{Ives2011} and interaction network \citep{Rafferty2013} data, optionally incorporating environmental, trait, and spatial structure\\
      \botrule
\end{tabular}}{}
\end{table*}
\subsection{Models}
\emph{pez} implements the \citet{Cavender-Bares2004} regression
framework for comparing matrices of species co-existence,
environmental, trait, and phylogenetic distance, and extends it by
including more distance metrics and measures of phylogenetic
signal. It also permits regression modelling of community
\citep{Ives2011} and interaction network data \citep{Rafferty2013}
using the Phylogenetic Generalised Linear Mixed Modelling (PGLMM)
framework. PGLMMs were the first truly statistical community
phylogenetic models to be developed, and permit sensitive tests of the
ecological and evolutionary structure of multiple communities.
\subsection{Simulations}
Developing new techniques requires the simulation and examination of
null eco-phylogenetic data, and \emph{pez} contains functions to
facilitate this.  The most advanced of these functions (\emph{scape})
is based on \citet{Helmus2012}, and simulates species' distributions
across environmental gradients. Species' environemntal tolerances and
range sizes can be simulated under Brownian, Ornstein-Uhlenbeck, and
disruptive models of trait evolution, and contrasting patterns of
phylogenetic structure at different spatial scales can be
generated. Another function (\emph{sim.meta.phy.comm}) permits less
sophisticated models of trait evolution and regional assembly, but
does not require an \emph{a priori} phylogeny, as it simulates
phylogeny, traits, and regional dynamics through evolutionary
time. Its sub-routines are available as separate functions within
\emph{pez} (table \ref{metricTable}), permitting rapid development and
prototyping of new methods and simulation studies.

\section*{Acknowledgements}
E.\ Lind and B.\ Waring provided feedback on the package and
manuscript.

\begin{thebibliography}{30}
\providecommand{\natexlab}[1]{#1}
\providecommand{\url}[1]{\texttt{#1}}
\providecommand{\urlprefix}{URL }

\bibitem[{Ackerly (2009)Ackerly, D.}]{Ackerly2009}
Ackerly, D. (2009) Conservatism and diversification of plant functional traits: Evolutionary rates versus phylogenetic signal \emph{Proceedings of the National Academy of
  Sciences} \textbf{106}, 19699-19706.

\bibitem[{Allen \emph{et~al.}(2009)Allen, Kon \& Bar-Yam}]{Allen2009}
Allen, B., Kon, M. \& Bar-Yam, Y. (2009) A new phylogenetic diversity measure
  generalizing the shannon index and its application to phyllostomid bats.
  \emph{The American Naturalist} \textbf{174}, 236--243.

\bibitem[{Cadotte \emph{et~al.}(2013)Cadotte, Albert \& Walker}]{Cadotte2013}
Cadotte, M., Albert, C.H. \& Walker, S.C. (2013) The ecology of differences:
  assessing community assembly with trait and evolutionary distances.
  \emph{Ecology Letters} \textbf{16}, 1234--1244.

\bibitem[{Cadotte \emph{et~al.}(2010)Cadotte, Davies, Regetz, Kembel, Cleland
  \& Oakley}]{Cadotte2010}
Cadotte, M.W., Davies, T.J., Regetz, J., Kembel, S.W., Cleland, E. \& Oakley,
  T.H. (2010) Phylogenetic diversity metrics for ecological communities:
  integrating species richness, abundance and evolutionary history.
  \emph{Ecology Letters} \textbf{13}, 96--105.

\bibitem[{Cavender-Bares \emph{et~al.}(2004)Cavender-Bares, Ackerly, Baum \&
  Bazzaz}]{Cavender-Bares2004}
Cavender-Bares, J., Ackerly, D.D., Baum, D.a. \& Bazzaz, F.a. (2004)
  {Phylogenetic overdispersion in Floridian oak communities}. \emph{The
  American Naturalist} \textbf{163}, 823--43.

\bibitem[{Faith(1992)}]{Faith1992}
Faith, D.P. (1992) Conservation evaluation and phylogenetic diversity.
  \emph{Biological Conservation} \textbf{61}, 1--10.

\bibitem[{Helmus \& Ives(2012)}]{Helmus2012}
Helmus, M.R. \& Ives, A.R. (2012) Phylogenetic diversity-area curves.
  \emph{Ecology} \textbf{93}, S31--S43.

\bibitem[{Ives \& Helmus (2010)Ives \& Helmus}]{Ives2010}
Ives, A.R. \& Helmus, M.R. (2010)
  Phylogenetic metrics of community similarity
  \emph{The American Naturalist} \textbf{176}, E128--E142.

\bibitem[{Ives \& Helmus (2011)Ives \& Helmus}]{Ives2011}Ives, A.R. \& Helmus, M.R. (2011) Generalized linear mixed models for phylogenetic analyses of community structure \emph{Ecological Monographs} \textbf{81}, 511--525

\bibitem[{Kembel(2009)}]{Kembel2009}
Kembel, S.W. (2009) {Disentangling niche and neutral influences on community
  assembly: assessing the performance of community phylogenetic structure
  tests}. \emph{Ecology Letters} \textbf{12}, 949--60.

\bibitem[{Kembel \emph{et~al.}(2010)Kembel, Cowan, Helmus, Cornwell, Morlon,
  Ackerly, Blomberg \& Webb}]{Kembel2010}
Kembel, S.W., Cowan, P.D., Helmus, M.R., Cornwell, W.K., Morlon, H., Ackerly,
  D.D., Blomberg, S.P. \& Webb, C.O. (2010) Picante: R tools for integrating
  phylogenies and ecology. \emph{Bioinformatics} \textbf{26}, 1463--1464.

\bibitem[{Letten \& Cornwell (in press)Letten \&
    Cornwell}]{Letten2014} Letten, A.D. \& Cornwell, W.K. (in press)
  Trees, branches and (square) roots: why evolutionary relatedness is
  not linearly related to functional distance. \emph{Methods in
    Ecology \& Evolution}.

\bibitem[{Ness \emph{et~al.}(2011)Ness, Rollinson \& Whitney}]{Ness2011}
Ness, J.H., Rollinson, E.J. \& Whitney, K.D. (2011) Phylogenetic distance can
  predict susceptibility to attack by natural enemies. \emph{Oikos}
  \textbf{120}, 1327--1334.

\bibitem[{Orme \emph{et~al.}(2013)Orme, Freckleton, Thomas, Petzoldt, Fritz,
  Isaac \& Pearse}]{Orme2013}
Orme, D., Freckleton, R., Thomas, G., Petzoldt, T., Fritz, S., Isaac, N. \&
  Pearse, W.D. (2013) \emph{caper: comparative analyses of phylogenetics and
  evolution in {R}}. \urlprefix\url{http://CRAN.R-project.org/package=caper}, r
  package version 0.5.2.

\bibitem[{Pagel(1999)}]{Pagel1999}
Pagel, M. (1999) {Inferring the historical patterns of biological evolution}.
  \emph{Nature} \textbf{401}, 877--884.

\bibitem[{Pearse \& Purvis(2013)}]{Pearse2013} Pearse, W.D. \& Purvis,
  A. (2013) phyloGenerator: an automated phylogeny generation tool for
  ecologists. \emph{Methods in Ecology and Evolution} \textbf{7},
  692--698.

\bibitem[{Pearse \emph{et~al.}(2014)Pearse, Cavender-Bares, Puvis \&
  Helmus}]{Pearse2014review}
Pearse, W.D., Cavender-Bares, J., Puvis, A. \& Helmus, M.R. (2014) Metrics and
  models of community phylogenetics. \emph{Modern Phylogenetic Comparative
  Methods and their Application in Evolutionary Biology---Concepts and
  Practice} (ed. L.Z. Garamszegi), Springer-Verlag, Berlin, Heidelberg.

\bibitem[{{R Core Team}(2015)}]{R2015}
{R Core Team} (2015) \emph{R: A language and environment for statistical
  computing}. R Foundation for Statistical Computing, Vienna, Austria.

\bibitem[{Rafferty \& Ives(2013)}]{Rafferty2013} Rafferty, N.E. \&
  Ives, A.R. (2013) Phylogenetic trait-based analyses of ecological
  networks. \emph{Ecology} \textbf{94}, 2321--2333.

\bibitem[{Webb(2000)}]{Webb2000}
Webb, C.O. (2000) {Exploring the phylogenetic structure of ecological
  communities: an example for rain forest trees}. \emph{The American
  Naturalist} \textbf{156}, 145--155.

\bibitem[{Webb \emph{et~al.}(2002)Webb, Ackerly, McPeek \& Donoghue}]{Webb2002}
Webb, C.O., Ackerly, D.D., McPeek, M.A. \& Donoghue, M.J. (2002) Phylogenies
  and community ecology. \emph{Annual Review of Ecology and Systematics}
  \textbf{33}, 475--505.

\end{thebibliography}
\end{document}