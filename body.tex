\documentclass{bioinfo}
\copyrightyear{2014}
\pubyear{2014}
\bibliographystyle{besjournals}
\begin{document}
\firstpage{1}

\title[\emph{pez}]{\emph{pez}: Phylogenetics for the Environmental
  Sciences} \author[Pearse \textit{et~al}]{William D.\
  Pearse\,$^{1}$\footnote{to whom correspondence should be addressed},
  Marc Cadotte\,$^{2}$, Jeannine Cavender-Bares\,$^1$, Caroline
  Tucker\,$^{3}$, Steve Walker\,$^{4}$ and Matthew R.\ Helmus\,$^5$}
\address{$^{1}$Department of Ecology, Evolution, and Behavior,
  University of Minnesota, 100 Ecology Building, 1987 Upper Buford
  Circle, Saint Paul, Minnesota, 55108, USA, $^{2}$Department of
  Biological Sciences, University of Toronto–Scarborough, 1265
  Military Trail, Scarborough, Ontario M1C 1A4, Canada
  $^{3}$Department of Ecology and Evolutionary Biology, University of
  Colorado, Boulder, CO, USA, $^{4}$Department of Mathematics and
  Statistics, McMaster University, Hamilton, Ontario L8S 4L8, Canada,
  $^{5}$Department of Animal Ecology, Vrije Universiteit, 1081 HV,
  Amsterdam, The Netherlands} \history{} \editor{}
\maketitle
\begin{abstract}
\section{Summary:}
\emph{pez} is an \emph{R} package that calculates $>$30 community
phylogenetic metrics, statistically models phylogenetic, trait, and
community data, and simulates community structure. It provides a
common programmatic standard for the management and manipulation of
these different data streams in \emph{R}.
\section{Availability:}
\emph{pez} is released under the GPL v3 open-source license, available
on the Internet from CRAN
(\href{http://cran.r-project.org}{http://cran.r-project.org}). The
package is under active development, and the authors welcome
contributions (see
\href{http://github.com/willpearse/pez}{http://github.com/willpearse/pez}).
\section{Contact:} \href{wdpearse@umn.edu}{William D.\ Pearse; wdpearse@umn.edu}
\end{abstract}
\section{Introduction}
Community phylogenetics (or eco-phylogenetics) is a sub-field of
ecology which links ecological phenomena with the evolutionary
processes that generate species and their traits
\citep[see][]{Webb2002}. This growing field has produced a number of
statistical tools and software code to implement them
\citep[\emph{e.g.},][]{Kembel2010}. This code is disparate and handles
data differently, making routine data analyses challenging. In many
cases, methods are not formally implemented in a software package and
are available only as supplementary materials to papers. Without
active (public) maintenance, these valuable techniques are effectively
lost to the scientific community.

\emph{pez} provides an \emph{R} \citep{R2014} class
(\emph{comparative.comm}) that integrates phylogenetic, community,
environmental, and trait data in a single object. Table
\ref{metricTable} gives an overview of its features. Using a
\emph{comparative.comm} object, one can calculate a large body of
phylogenetic biodiversity metrics, many of which were previously
unavailable in \emph{R}. These metrics are grouped according to the
framework outlined in \citet{Pearse2014review}, easing the
interpretation and comparison of different aspects of phylogenetic
biodiversity. Further, \emph{pez} implements and extends the
regression framework presented by \citep{Cavender-Bares2004}, and
provides functions to simulate phylogenetic and ecological data. It is
our hope that \emph{pez} will provide a universal, easy to use, and
extendable \emph{R} framework for eco-phylogenetic analysis.
\section{Description}
\subsection{Data manipulation and storage}
\emph{pez} extends the data manipulation and import/export features of
\emph{R}, and contains a unified class to contain all eco-phyogenetic
data. It simplifies the process of ensuring community, trait,
environmental, and phylogenetic data are compatible by, for example,
allowing the simultaneous removal of species from all datasets
simultaneously. \emph{pez} is compatible with \emph{caper}'s
\citep{Orme2013} \emph{comparative.data} class, aiding comparative
analysis of species trait data.

\subsection{Metrics}
Following the classification of \citet{Pearse2014review}, \emph{pez}
facilitates the calculation and comparison of a number of metrics by
grouping them into four categories: \emph{shape}, \emph{evenness},
\emph{dispersion}, and \emph{dissimilarity}. Shape metrics measure the
structure of a community phylogeny, while evenness metrics
additionally incorporate species abundances. Dispersion metrics are
used to examine whether phylogenetic biodiversity in an assemblage
differs from the expectation of random assembly from a given set of
species, a species pool. Finally, dissimilarity metrics measure the
pairwise difference in phylogenetic biodiversity between
assemblages. The \emph{traitgram} framework
\citep{Ackerly2009,Cadotte2013}, which directly compares phylogenetic
and functional trait community structure, is also
implemented. Speeding and easing the comparison of community
phylogenetic metrics is important, since different metrics can reveal
separate aspects eco-phylogenetic structure \citep{Cadotte2010}.

\begin{table*}
  \processtable{Overview of some of the functions available in \emph{pez}.\label{metricTable}}
  {\begin{tabular}{p{2.5cm} p{15cm}}\toprule
      Function(s) & Description\\\midrule
      \emph{comparative.comm} & Stores community, phylogenetic, environmental, and species trait data\\
      \emph{shape}, \emph{evenness}, \emph{dispersion}, \emph{dissimilarity}


      \emph{shape} & Calculates \emph{PSV} \citep{Helmus2007}, \emph{PSR} \citep{Helmus2007}, \emph{PD} \& \emph{MPD} \citep{Faith1992}, Colless' Index \citep{Colless1982}, $\gamma$ \citep{Pybus2000}, $\Delta$ \citep{Warwick1995}, $E_{ED}$ \& $H_{ed}$ \citep{Cadotte2010}, phylo-eigenvectors \citep{Diniz-Filho2011}\\
      \emph{evenness} & Calculates $\Delta$ \citep{Warwick1995}, Phylogenetic Entropy \citep{Allen2009}, \emph{PAE}, \emph{IAC}, $H_{aed}$, \& $E_{aed}$ \citep{Cadotte2010}, Rao's quadratic entropy \citep{Rao1982a}, $\lambda$ \citep{Pagel1999}, $\delta$ \citep{Pagel1999}, $\kappa$ \citep{Pagel1999}\\
      \emph{dispersion} & Calculates $SES_{MPD}$/\emph{NRI} \citep{Webb2002,Kembel2009}, $SES_{MNTD}/\emph{NTI}$ \citep{Webb2002,Kembel2009}, \emph{INND} \citep{Ness2011}, \emph{D} \citep{Fritz2010}\\
      \emph{dissimilarity} &  Calculates UniFrac \citep{Lozupone2005}, \emph{PCD} \citep{Ives2010}, PhyloSor \citep{Bryant2008}, Rao's Q \citep{Rao1982a}\\
      \emph{trait.asm} & Calculates optimal \emph{traitgram} that explains community data, following \citep{Ackerly2009,Cadotte2013}\\
      \emph{fingerprint.regression} & Compares phylogenetic, community co-existence, and trait similarity matrices using Mantel tests, quantile regressions, or linear models, following \citep{Cavender-Bares2004} \\
      \emph{scape} & Simulates community phylogenetic structure across a landscape, simulating phylogenetic repulsion, attraction, niche width, and range size \citep{Helmus2012}\\
      \emph{sim.meta.phy.comm} & Simulates a phylogeny with associated species' traits under iterated regional assembly and dispersal; sub-components available as separate functions  (\emph{sim.bd.xxx}, and \emph{sim.meta.xxx}).\\
      \botrule
\end{tabular}}{}
\end{table*}
\subsection{Models and Simulations}
\emph{pez} implements the \citet{Cavender-Bares2004} regression
framework for comparing species co-existence, environmental, trait,
and phylogenetic distance matrices, and extends it by including more
distance metrics and measures of phylogenetic signal. Developing new
techniques requires the simulation and examination of null
communities, and \emph{pez} contains functions to facilitate this.
The most advanced of these functions (\emph{scape}) is based on
\citet{Helmus2012}, and simulates species' distributions across
environmental gradients. Species' environemntal tolerances and range
sizes can be simulated under Brownian, Ornstein-Uhlenbeck, and
disruptive models of trait evolution, and contrasting patterns of
phylogenetic structure at different spatial scales can be
generated. Another function (\emph{sim.meta.phy.comm}) permits less
sophisticated models of trait evolution and regional assembly, but
does not require an \emph{a priori} phylogeny, and simulates
phylogeny, traits, and regional dynamics through evolutionary
time. Its sub-routines are available as separate functions within
\emph{pez} (table \ref{metricTable}), permitting rapid development and
prototyping of new methods and simulation studies.

\section*{Acknowledgements}
A.\ Purvis contributed to the metric framework according to which this
package's metrics are organised.
\paragraph{Funding\textcolon} WDP was funded by XXX, MRH was funded by
the Netherland Organisation for Scientific Research (858.14.040); both
were funded by SESYNC (XXX).

\begin{thebibliography}{30}
\providecommand{\natexlab}[1]{#1}
\providecommand{\url}[1]{\texttt{#1}}
\providecommand{\urlprefix}{URL }

\bibitem[{Ackerly (2009)Ackerly, D.}]{Ackerly2009}
Ackerly, D. (2009) Conservatism and diversification of plant functional traits: Evolutionary rates versus phylogenetic signal \emph{Proceedings of the National Academy of
  Sciences} \textbf{106}, 19699-19706.

\bibitem[{Allen \emph{et~al.}(2009)Allen, Kon \& Bar-Yam}]{Allen2009}
Allen, B., Kon, M. \& Bar-Yam, Y. (2009) A new phylogenetic diversity measure
  generalizing the shannon index and its application to phyllostomid bats.
  \emph{The American Naturalist} \textbf{174}, 236--243.

\bibitem[{Bryant \emph{et~al.}(2008)Bryant, Lamanna, Morlon, Kerkhoff, Enquist
  \& Green}]{Bryant2008}
Bryant, J.A., Lamanna, C., Morlon, H., Kerkhoff, A.J., Enquist, B.J. \& Green,
  J.L. (2008) Microbes on mountainsides: contrasting elevational patterns of
  bacterial and plant diversity. \emph{Proceedings of the National Academy of
  Sciences} \textbf{105}, 11505--11511.

\bibitem[{Cadotte \emph{et~al.}(2013)Cadotte, Albert \& Walker}]{Cadotte2013}
Cadotte, M., Albert, C.H. \& Walker, S.C. (2013) The ecology of differences:
  assessing community assembly with trait and evolutionary distances.
  \emph{Ecology Letters} \textbf{16}, 1234--1244.

\bibitem[{Cadotte \emph{et~al.}(2010)Cadotte, Davies, Regetz, Kembel, Cleland
  \& Oakley}]{Cadotte2010}
Cadotte, M.W., Davies, T.J., Regetz, J., Kembel, S.W., Cleland, E. \& Oakley,
  T.H. (2010) Phylogenetic diversity metrics for ecological communities:
  integrating species richness, abundance and evolutionary history.
  \emph{Ecology Letters} \textbf{13}, 96--105.

\bibitem[{Cavender-Bares \emph{et~al.}(2004)Cavender-Bares, Ackerly, Baum \&
  Bazzaz}]{Cavender-Bares2004}
Cavender-Bares, J., Ackerly, D.D., Baum, D.a. \& Bazzaz, F.a. (2004)
  {Phylogenetic overdispersion in Floridian oak communities}. \emph{The
  American Naturalist} \textbf{163}, 823--43.

\bibitem[{Colless(1982)}]{Colless1982}
Colless, D.H. (1982) Review of phylogenetics: the theory and practice of
  phylogenetic systematics. \emph{Systematic Zoology} \textbf{31}, 100--104.

\bibitem[{Diniz-Filho \emph{et~al.}(2011)Diniz-Filho, Cianciaruso, Rangel \&
  Bini}]{Diniz-Filho2011}
Diniz-Filho, J.A.F., Cianciaruso, M.V., Rangel, T.F. \& Bini, L.M. (2011)
  Eigenvector estimation of phylogenetic and functional diversity.
  \emph{Functional Ecology} \textbf{25}, 735--744.

\bibitem[{Faith(1992)}]{Faith1992}
Faith, D.P. (1992) Conservation evaluation and phylogenetic diversity.
  \emph{Biological Conservation} \textbf{61}, 1--10.

\bibitem[{Fritz \& Purvis(2010)}]{Fritz2010}
Fritz, S.A. \& Purvis, A. (2010) selectivity in mammalian extinction risk and
  threat types: a new measure of phylogenetic signal strength in binary traits.
  \emph{Conservation Biology} \textbf{24}, 1042--1051.

\bibitem[{Helmus \emph{et~al.}(2007)}]{Helmus2007} Helmus, Matthew
  R. and Bland, Thomas J. and Williams, Christopher K. and Ives,
  Anthony R. (2007) Phylogenetic metrics of biodiversity \emph{The
    American Naturalist} \textbf{169}, E68--E83.

\bibitem[{Helmus \& Ives(2012)}]{Helmus2012}
Helmus, M.R. \& Ives, A.R. (2012) Phylogenetic diversity-area curves.
  \emph{Ecology} \textbf{93}, S31--S43.

\bibitem[{Ives \& Helmus (2010)Ives \& Helmus}]{Ives2010}
Ives, A.R. \& Helmus, M.R. (2010)
  Phylogenetic metrics of community similarity
  \emph{The American Naturalist} \textbf{176}, E128--E142.

\bibitem[{Kembel(2009)}]{Kembel2009}
Kembel, S.W. (2009) {Disentangling niche and neutral influences on community
  assembly: assessing the performance of community phylogenetic structure
  tests}. \emph{Ecology Letters} \textbf{12}, 949--60.

\bibitem[{Kembel \emph{et~al.}(2010)Kembel, Cowan, Helmus, Cornwell, Morlon,
  Ackerly, Blomberg \& Webb}]{Kembel2010}
Kembel, S.W., Cowan, P.D., Helmus, M.R., Cornwell, W.K., Morlon, H., Ackerly,
  D.D., Blomberg, S.P. \& Webb, C.O. (2010) Picante: R tools for integrating
  phylogenies and ecology. \emph{Bioinformatics} \textbf{26}, 1463--1464.

\bibitem[{Lozupone \& Knight(2005)}]{Lozupone2005}
Lozupone, C. \& Knight, R. (2005) {UniFrac}: a new phylogenetic method for
  comparing microbial communities. \emph{Applied and Environmental
  Microbiology} \textbf{71}, 8228--8235.

\bibitem[{Ness \emph{et~al.}(2011)Ness, Rollinson \& Whitney}]{Ness2011}
Ness, J.H., Rollinson, E.J. \& Whitney, K.D. (2011) Phylogenetic distance can
  predict susceptibility to attack by natural enemies. \emph{Oikos}
  \textbf{120}, 1327--1334.

\bibitem[{Orme \emph{et~al.}(2013)Orme, Freckleton, Thomas, Petzoldt, Fritz,
  Isaac \& Pearse}]{Orme2013}
Orme, D., Freckleton, R., Thomas, G., Petzoldt, T., Fritz, S., Isaac, N. \&
  Pearse, W.D. (2013) \emph{caper: comparative analyses of phylogenetics and
  evolution in {R}}. \urlprefix\url{http://CRAN.R-project.org/package=caper}, r
  package version 0.5.2.

\bibitem[{Pagel(1999)}]{Pagel1999}
Pagel, M. (1999) {Inferring the historical patterns of biological evolution}.
  \emph{Nature} \textbf{401}, 877--884.

\bibitem[{Pearse \emph{et~al.}(2014)Pearse, Cavender-Bares, Puvis \&
  Helmus}]{Pearse2014review}
Pearse, W.D., Cavender-Bares, J., Puvis, A. \& Helmus, M.R. (2014) Metrics and
  models of community phylogenetics. \emph{Modern Phylogenetic Comparative
  Methods and their Application in Evolutionary Biology---Concepts and
  Practice} (ed. L.Z. Garamszegi), Springer-Verlag, Berlin, Heidelberg.

\bibitem[{Pybus \& Harvey(2000)}]{Pybus2000}
Pybus, O. \& Harvey, P. (2000) Testing macro-evolutionary models using
  incomplete molecular phylogenies. \emph{Proceedings of the Royal Society B:
  Biological Sciences} \textbf{267}, 2267--2272.

\bibitem[{{R Core Team}(2014)}]{R2014}
{R Core Team} (2014) \emph{R: A language and environment for statistical
  computing}. R Foundation for Statistical Computing, Vienna, Austria.

\bibitem[{Rao(1982)}]{Rao1982a}
Rao, C. (1982) Diversity and dissimilarity coefficients: a unified approach.
  \emph{Theoretical Population Biology} \textbf{21}, 24--43.

\bibitem[{Warwick \& Clarke(1995)}]{Warwick1995}
Warwick, R.M. \& Clarke, K.R. (1995) New `biodiversity' measures reveal a
  decrease in taxonomic distinctness with increasing stress. \emph{Marine
  Ecology Progress Series} \textbf{129}.

\bibitem[{Webb(2000)}]{Webb2000}
Webb, C.O. (2000) {Exploring the phylogenetic structure of ecological
  communities: an example for rain forest trees}. \emph{The American
  Naturalist} \textbf{156}, 145--155.

\bibitem[{Webb \emph{et~al.}(2002)Webb, Ackerly, McPeek \& Donoghue}]{Webb2002}
Webb, C.O., Ackerly, D.D., McPeek, M.A. \& Donoghue, M.J. (2002) Phylogenies
  and community ecology. \emph{Annual Review of Ecology and Systematics}
  \textbf{33}, 475--505.

\end{thebibliography}
\end{document}