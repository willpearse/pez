% add the sqrt stuff and trait stuff into the metrics description
% section, highlighting novelty of it.
% --> probably have to remove the table
% - point out how it extends picante in the metrics section.
% - remove "extends features of R"
% picante comparison
% add package refs
% "extends R"
% PGLMM structure thing
% data structure 
\documentclass{bioinfo}
\copyrightyear{2015}
\pubyear{2015}
\begin{document}
\firstpage{1}

\title[\emph{pez}]{\emph{pez}: Phylogenetics for the Environmental
  Sciences} \author[Pearse \textit{et~al}]{William D.\
  Pearse\,$^{1,2,3}$\footnote{to whom correspondence should be
    addressed}, Marc Cadotte\,$^{4}$, Jeannine Cavender-Bares\,$^1$,
  Anthony R.\ Ives$^5$, Caroline Tucker\,$^6$, Steve Walker\,$^7$ and
  Matthew R.\ Helmus\,$^8$} \address{$^1$Department of Ecology,
  Evolution, and Behavior, University of Minnesota, 100 Ecology
  Building, 1987 Upper Buford Circle, Saint Paul, Minnesota, USA,
  $^2$Department of Biology, McGill University, 1205 Avenue Docteur
  Penfield, Montr\'{e}al, Qu\'{e}bec, Canada, $^3$D\'{e}partement des
  Sciences Biologiques, Universit\'{e} du Qu\'{e}bec \`{a}
  Montr\'{e}al, Succursale Centre-ville, Montr\'{e}al, Qu\'{e}bec,
  Canada, $^4$Department of Biological Sciences, University of
  Toronto–Scarborough, 1265 Military Trail, Scarborough, Ontario,
  Canada, $^5$Department of Zoology, University of Wisconsin, Madison,
  Wisconsin, USA $^6$Department of Ecology and Evolutionary Biology,
  University of Colorado, Boulder, CO, USA, $^7$Department of
  Mathematics and Statistics, McMaster University, Hamilton, Ontario,
  Canada, $^8$Department of Animal Ecology, Vrije Universiteit, 1081
  HV, Amsterdam, The Netherlands} \history{} \editor{}
\maketitle
\begin{abstract}
\section{Summary:}
\emph{pez} is an \emph{R} package that permits the measurement,
modelling, and simulation of phylogenetic structure in ecological
data. While \emph{pez} permits that application of many methods for
the first time, it also synthesises existing data structures and
analyses into a single, coherent framework.
\section{Availability:}
\emph{pez} is released under the GPL v3 open-source license, available
on the Internet from CRAN
(\href{http://cran.r-project.org}{http://cran.r-project.org}). The
package is under active development, and the authors welcome
contributions (see
\href{http://github.com/willpearse/pez}{http://github.com/willpearse/pez}).
\section{Contact:} \href{will.pearse@gmail.com}{William D.\ Pearse; will.pearse@gmail.com}
\end{abstract}
\section{Introduction}
Community phylogenetics (or eco-phylogenetics) combines ecology and
evolutionary biology, linking ecological phenomena with the
evolutionary processes that generate species and their traits
\citep[see][]{Webb2002}. This growing field has produced a number of
statistical tools and software code to implement them
\citep[\emph{e.g.},][]{Kembel2010}, but the field's code is disparate
and handles data differently, making routine data analyses
challenging. In many cases, methods are not formally implemented in a
software package and are available only as supplementary materials to
papers. Without active (public) maintenance, these valuable techniques
are effectively lost to the scientific community.

\emph{pez} provides an \emph{R} \citep{R2015} class
(\emph{comparative.comm}) that integrates phylogenetic, community,
environmental, and trait data in a single object. Using a
\emph{comparative.comm} object, one can calculate a large body of
phylogenetic biodiversity metrics, some of which were previously
unavailable in \emph{R}. These metrics can be calculated using
phylogenetic and functional trait data, strengthening links between
eco-phylogenetics and functional trait ecology. \emph{pez} also
implements several previously unavailable modelling frameworks
\citep{Cavender-Bares2004,Ives2011,Rafferty2013}, helping users move
beyond describing differences between phylogenetic and functional
trait data and instead jointly modelling them. Combined with new
functions for the simulation of eco-phylogenetic data, it is our hope
that \emph{pez} will provide an easy to use framework for existing
eco-phylogenetic methods, and support the development of the next
generation of methods and approaches.
\section{Description}
\subsection{Data manipulation and storage}
\emph{pez} provides a unified class to contain all eco-phylogenetic
data, and provides wrapper functions to safely and conveniently
manipulate the species and site composition of community, trait,
environmental, and phylogenetic data. \emph{pez} integrates and makes
use of much existing \emph{R} code
\citep{Bortolussi2012,Genz2013,Koenker2015,Labierte2014,Oksanen2015,Paradis2004},
and its \emph{comparative.comm} class is directly compatible with all
\emph{caper} \citep{Orme2013} code, easing comparative analysis of
species trait data.

\subsection{Metrics}
Following the classification of \citet{Pearse2014review}, \emph{pez}
eases the calculation and comparison of over 30 metrics by grouping
them into four categories: \emph{shape}, \emph{evenness},
\emph{dispersion}, and \emph{dissimilarity}. Shape metrics measure the
structure of a community phylogeny, while evenness metrics
additionally incorporate species abundances. Dispersion metrics are
used to examine whether phylogenetic biodiversity in an assemblage
differs from the expectation of random assembly from a given set of
species---a species pool. Finally, dissimilarity metrics measure the
pairwise difference in phylogenetic biodiversity between
assemblages. This classification links directly onto the kind of data
the investigator has at hand; assemblage occupancy (\emph{shape}),
assemblage abundance (\emph{evenness}), an assemblage and a
hypothesised source pool (\emph{dispersion}), and two assemblages to
be compared (\emph{dissimilarity}).

Other packages \citep[notably][]{Kembel2010} allow for the calculation
of many (but not all) of the metrics contained within
\emph{pez}. \emph{pez} unifies these metrics under a common interface,
and extends them to use functional trait, distance matrix, and
transformed phylogenetic \citep[\emph{sensu}][]{Letten2014}
data. Indeed, \emph{pez} provides the first quantitative
implementation of the \emph{traitgram} framework
\citep{Ackerly2009,Cadotte2013}, allowing the direct comparison of the
extent of phylogenetic and functional trait community
structure. Speeding and easing the comparison of community
phylogenetic metrics is important, since different metrics reveal
separate aspects of eco-phylogenetic structure
\citep{Cadotte2010}. \emph{pez} also contains a flexible metric
estimation and null-model generation suite (the \emph{generic.metric}
family), that can be used to easily compare different metrics and
perhaps generate new \emph{dispersion} metrics. New null models based
around species' trait values (using \emph{trait.asm}), and expected
mean pairwise phylogenetic and trait distances (using
\emph{ConDivSim}) can also be simulated.
\subsection{Models}
\emph{pez} implements the \citet{Cavender-Bares2004} regression
framework (\emph{fingerprint.regression}), and extends it to include
more metrics of trait evolution. While community phylogenetics can be
criticised for relying on the assumption of niche conservatism to
explain ecological assembly, this early (yet previously unavailable)
approach does not rely on niche conservatism. Instead, it regresses
the correlation between species' trait similarity and observed degree
of co-occurrence against summary statistics of the evolution of those
traits. The approach goes beyond simply describing the phylogenetic or
ecological structure of an assemblage, and instead draws links between
how traits have evolved and how they play-out in their present-day
ecological context.

The \emph{pglmm} family of functions permit regression modelling of
community \citep{Ives2011} and interaction network data
\citep{Rafferty2013} using the Phylogenetic Generalised Linear Mixed
Modelling (PGLMM) framework. PGLMMs were the first truly statistical
community phylogenetic models to be developed, and permit sensitive
tests of the ecological and evolutionary structure of multiple
communities. By fitting random effect terms that account for
phylogenetic co-variance, the user can statistically model how likely
species are to (co-) occur as a function of both their traits and
environmental conditions.
\subsection{Simulations}
Developing new techniques requires the simulation and examination of
null eco-phylogenetic data, and \emph{pez} contains functions to
facilitate this. The \emph{scape} family of functions
\citep[following][]{Helmus2012} simulate species' distributions across
environmental gradients. Species' environmental tolerances and range
sizes can be simulated under Brownian, Ornstein-Uhlenbeck, and
disruptive models of trait evolution, and contrasting patterns of
phylogenetic structure at different spatial scales can be
generated. The \emph{sim.meta.phy.comm} family of functions permit
simultaneous simulation of phylogeny, trait evolution, and assemblage
structure across a simulated landscape. Abundances are proportional to
the similarity between each individual of a given species' fit to the
grid cell it is within. While (unlike \emph{scape}) these functions do
not require the user to specify a phylogeny, they do not simulate
species interactions. However, its sub-routines (and those of the
related \emph{phy.sim} family) are available as separate functions
within \emph{pez}, permitting rapid development and prototyping of new
methods and simulation studies.
\section*{Acknowledgements}
E.\ Lind, B.\ Waring, two anonymous reviewers, and Matt Pennell
provided feedback on the package and manuscript.

\begin{thebibliography}{30}
\providecommand{\natexlab}[1]{#1}
\providecommand{\url}[1]{\texttt{#1}}
\providecommand{\urlprefix}{URL }

\bibitem[{Ackerly (2009)Ackerly, D.}]{Ackerly2009}
Ackerly, D. (2009) Conservatism and diversification of plant functional traits: Evolutionary rates versus phylogenetic signal \emph{Proceedings of the National Academy of
  Sciences} \textbf{106}, 19699-19706.

\bibitem[{Bortolussi \emph{et~al.}(2012)Bortolussi, Durand, Blum \&
    Francois}]{Bortolussi2012} Bortolussi, N., Durand, E., Blum, M. \&
  Francois, O. (2012) \emph{apTreeshape: Analyses of Phylogenetic
    Treeshape}. R package version
  1.4-5. \urlprefix\url{http://CRAN.R-project.org/package=apTreeshape}

\bibitem[{Cadotte \emph{et~al.}(2013)Cadotte, Albert \& Walker}]{Cadotte2013}
Cadotte, M., Albert, C.H. \& Walker, S.C. (2013) The ecology of differences:
  assessing community assembly with trait and evolutionary distances.
  \emph{Ecology Letters} \textbf{16}, 1234--1244.

\bibitem[{Cadotte \emph{et~al.}(2010)Cadotte, Davies, Regetz, Kembel, Cleland
  \& Oakley}]{Cadotte2010}
Cadotte, M.W., Davies, T.J., Regetz, J., Kembel, S.W., Cleland, E. \& Oakley,
  T.H. (2010) Phylogenetic diversity metrics for ecological communities:
  integrating species richness, abundance and evolutionary history.
  \emph{Ecology Letters} \textbf{13}, 96--105.

\bibitem[{Cavender-Bares \emph{et~al.}(2004)Cavender-Bares, Ackerly, Baum \&
  Bazzaz}]{Cavender-Bares2004}
Cavender-Bares, J., Ackerly, D.D., Baum, D.a. \& Bazzaz, F.a. (2004)
  {Phylogenetic overdispersion in Floridian oak communities}. \emph{The
  American Naturalist} \textbf{163}, 823--43.

\bibitem[{Dray \& DuFour (2007)Dray \& Dufor}]{Dray2007} Dray, S. \&
  Dufour, A.B. (2007) The ade4 package: implementing the duality
  diagram for ecologists. Journal of Statistical Software. 22(4):
  1--20.

\bibitem[{Genz \emph{et~al.}(2013)Genz, Bretz, Miwa, Mi, Leisch,
    Scheipl, \& Hothorn}]{Genz2013} Genz, A., Bretz, F., Miwa, T., Mi,
  X., Leisch, F., Scheipl, F., Hothorn, T. (2013) \emph{mvtnorm:
    Multivariate Normal and t
    Distributions}. \urlprefix\url{http://CRAN.R-project.org/package=mvtnorm},
  R package version 1.0-2.

\bibitem[{Helmus \& Ives(2012)}]{Helmus2012}
Helmus, M.R. \& Ives, A.R. (2012) Phylogenetic diversity-area curves.
  \emph{Ecology} \textbf{93}, S31--S43.

\bibitem[{Ives \& Helmus(2011)}]{Ives2011}Ives, A.R. \&
  Helmus, M.R. (2011) Generalized linear mixed models for phylogenetic
  analyses of community structure \emph{Ecological Monographs}
  \textbf{81}, 511--525

\bibitem[{Kembel \emph{et~al.}(2010)Kembel, Cowan, Helmus, Cornwell, Morlon,
  Ackerly, Blomberg \& Webb}]{Kembel2010}
Kembel, S.W., Cowan, P.D., Helmus, M.R., Cornwell, W.K., Morlon, H., Ackerly,
  D.D., Blomberg, S.P. \& Webb, C.O. (2010) Picante: R tools for integrating
  phylogenies and ecology. \emph{Bioinformatics} \textbf{26}, 1463--1464.

\bibitem[{Koenker (2015)}]{Koenker2015} Koenker, R.\ (2015)
  \emph{quantreg: Quantile
    Regression}. \urlprefix\url{http://CRAN.R-project.org/package=quantreg},
  R package version 5.11.

\bibitem[{Labiert\'{e} \emph{et~al.}(2014)}]{Labierte2014} Labiert\'{e}, E., Legendre,
  P. \& Shipley, B. (2014) \emph{FD: measuring functional diversity
    from multiple traits, and other tools for functional
    ecology}. \urlprefix\url{http://CRAN.R-project.org/package=FD}, R
  package version 1.0-12.

\bibitem[{Letten \& Cornwell (in press)Letten \&
    Cornwell}]{Letten2014} Letten, A.D. \& Cornwell, W.K. (in press)
  Trees, branches and (square) roots: why evolutionary relatedness is
  not linearly related to functional distance. \emph{Methods in
    Ecology \& Evolution}.

\bibitem[{Oksanen \emph{et~al.}(2015)Oksanen, Blanchet, Kindt,
  Legendre, Minchin, O'Hara, Simpson, Solymos, Stevens,
  Wagner}]{Oksanen2015} Oksanen, J., Guillaume, Blanchet, F.G., Kindt,
R., Legendre, P., Minchin, P.R., O'Hara, R.B., Simpson, G.L., Solymos,
P., Stevens, M.H.H. \& Wagner, H. (2015) \emph{vegan: Community Ecology
  Package}. \urlprefix\url{http://CRAN.R-project.org/package=vegan}, R
package version 2.2-1.

\bibitem[{Orme \emph{et~al.}(2013)Orme, Freckleton, Thomas, Petzoldt, Fritz,
  Isaac \& Pearse}]{Orme2013}
Orme, D., Freckleton, R., Thomas, G., Petzoldt, T., Fritz, S., Isaac, N. \&
  Pearse, W.D. (2013) \emph{caper: comparative analyses of phylogenetics and
  evolution in {R}}. \urlprefix\url{http://CRAN.R-project.org/package=caper}, r
  package version 0.5.2.

\bibitem[{Paradis \emph{et~al.}(2004)Paradis, Claude \&
    Strimmer}]{Paradis2004} Paradis, E., Claudem, J. \& Strimmer, K.
  (2004) APE: analyses of phylogenetics and evolution in R
  language. \emph{Bioinformatics} \textbf{20} 289--290.

\bibitem[{Pearse \& Purvis(2013)}]{Pearse2013} Pearse, W.D. \& Purvis,
  A. (2013) phyloGenerator: an automated phylogeny generation tool for
  ecologists. \emph{Methods in Ecology and Evolution} \textbf{7},
  692--698.

\bibitem[{Pearse \emph{et~al.}(2014)Pearse, Cavender-Bares, Puvis \&
  Helmus}]{Pearse2014review}
Pearse, W.D., Cavender-Bares, J., Puvis, A. \& Helmus, M.R. (2014) Metrics and
  models of community phylogenetics. \emph{Modern Phylogenetic Comparative
  Methods and their Application in Evolutionary Biology---Concepts and
  Practice} (ed. L.Z. Garamszegi), Springer-Verlag, Berlin, Heidelberg.

\bibitem[{{R Core Team}(2015)}]{R2015}
{R Core Team} (2015) \emph{R: A language and environment for statistical
  computing}. R Foundation for Statistical Computing, Vienna, Austria.

\bibitem[{Rafferty \& Ives(2013)}]{Rafferty2013} Rafferty, N.E. \&
  Ives, A.R. (2013) Phylogenetic trait-based analyses of ecological
  networks. \emph{Ecology} \textbf{94}, 2321--2333.

\bibitem[{Webb \emph{et~al.}(2002)Webb, Ackerly, McPeek \& Donoghue}]{Webb2002}
Webb, C.O., Ackerly, D.D., McPeek, M.A. \& Donoghue, M.J. (2002) Phylogenies
  and community ecology. \emph{Annual Review of Ecology and Systematics}
  \textbf{33}, 475--505.
\end{thebibliography}
\end{document}
%%% Local Variables:
%%% mode: latex
%%% TeX-master: t
%%% End:
