\documentclass[fontsize=12pt, paper=a4]{scrlttr2}
\usepackage[english]{babel}
\usepackage[parfill]{parskip}  
\usepackage{geometry}
\usepackage{url}

\setkomavar{fromname}{Will Pearse}
\setkomavar{fromaddress}{Department of Ecology, Evolution and Behavior\\University of Minnesota\\St. Paul, MN 55108, USA}

\setkomavar{signature}{Will Pearse}
\renewcommand{\raggedsignature}{\raggedright} % make the signature ragged right
\makeatletter
\@setplength{refvpos}{\useplength{toaddrvpos}}
\makeatother

\setkomavar{subject}{Submission of ``\emph{pez}: Phylogenetics for the Environmental Sciences}
\KOMAoptions{fromalign=left, fromlogo=false, addrfield=no,
backaddress=no,  subject=left,foldmarks=no,
fromphone=no,fromemail=true,fromalign=right,pagenumber=no}
\setkomavar{fromemail}{wdpearse@umn.edu}
\begin{document}
\begin{letter}

\opening{To whom it may concern,}\enlargethispage{8\baselineskip}

I am writing to ask you to consider the manuscript `\emph{pez:
  Phylogenetics for the Environmental Sciences}', for publication in
\emph{Bioinformatics}. The manuscript describes an \emph{R} package
(\emph{pez}) that integrates and facilitates the analysis and
simulation of evolutionary and ecological data.

An ever-increasing number of ecologists test hypotheses using
phylogenetic data, but there is no unified data structure or interface
that integrates diverse data-types. Moreover, there are a number of
statistical methods that, despite their popularity, have not been
formally released as software packages and as such are patchily and
inconsistently implemened. \emph{pez} solves the first of these
problems and goes some way to addressing the second, while also
integrating a number of existing methods under a common interface. It
is my hope that \emph{pez} will more tightly integrate the
computational development of the field of eco-phylogenetics, ensuring
the fragmentation I believe \emph{pez} to have reversed will not
recur.

I feel this manuscript is best-suited to \emph{Bioinformatics} because
many conceptually related \emph{R} packages (\emph{e.g.},
\emph{picante} and \emph{geiger}) have been published there. The
package has been developed using modern, open software development
techniques, which I feel your journal is well-placed to
review. Potentially suitable reviewers include Scott Chamberlain
(XXX), Stephen Kembell (XXX), and David Orme
(\url{d.orme@imperial.ac.uk}, who have expertise implementing
statistical methods similar to those in \emph{pez}.

\closing{Thank you for your time, and I look forward to hearing from
  you,\vspace{20pt}}

\end{letter}
\end{document}